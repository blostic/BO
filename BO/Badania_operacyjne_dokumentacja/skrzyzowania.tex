\documentclass{article}
\usepackage{graphicx}
\usepackage{cgw06}
\usepackage{epsf}
\usepackage{polski}
\usepackage[T1]{fontenc}
\usepackage[utf8]{inputenc}
\usepackage{hyperref}
\usepackage{xcolor}
\usepackage[all]{hypcap}
\usepackage{color}
\usepackage{mdframed}
\usepackage{titletoc}
\usepackage{listings}
\usepackage{mathtools}

\definecolor{dark-red}{rgb}{0.5,0,0}
\definecolor{dark-green}{rgb}{0,0.5,0}
\definecolor{dark-blue}{rgb}{0,0,0.5}
\hypersetup{
    colorlinks,
    linkcolor={dark-blue},
    urlcolor={dark-blue},
    citecolor={dark-green}
}

\lstdefinestyle{customc}{
  belowcaptionskip=1\baselineskip,
  breaklines=true,
  frame=L,
  xleftmargin=\parindent,
  language=Pascal,
  showstringspaces=false,
  basicstyle=\footnotesize\ttfamily,
  keywordstyle=\bfseries\color{green!40!black},
  commentstyle=\itshape\color{purple!40!black},
  identifierstyle=\color{blue},
  stringstyle=\color{orange},
}

\lstset{escapechar=@,style=customc}
\title {Skrzyżowania ++}
    
\author{Piotr Skibiak, Tomasz Kwiecień, Marcin Nowak, Ksawery Głaz, Paweł Łabno}
\institute{AGH, Wydział IEiT, Informatyka}

\begin{document}
\maketitle

\begin{abstract}
Projekt ma na celu skrócenie oczekiwania samochodów na skrzyżowaniach, poprzez zastosowanie optymalnego ustawienia czasu świateł. Optymalne (albo prawie optymalne) ustawienie świateł wyznaczane jest poprzez użycie jednego z algorytmów stadnych - algorytmu kukułki.   
\end{abstract}

\newpage
\renewcommand*\contentsname{Spis Treści}
\tableofcontents
\setcounter{tocdepth}{3}

\newpage

\section{Przedstawienie problemu}



\section{Cele projektu}
    Projekt ma na celu umożliwienie użytkownikowi znalezienie optymalnego ustawienia czasów świecenia świateł (światła zielonego i czerwonego) w sieci skrzyżowań ustalonych przez użytkownika. Aplikacja powinna dać użytkownikowi możliwość ręcznego wprowadzenia sieci skrzyżowań, zapisu konfiguracji do pliku, lub wczytania wcześniej zapisanej. 

\section{Wstępne założenia}
\subsection{Środowisko Implementacji}
    Aby pogodzić potrzebę szybkiego wykonywania symulacji, oraz zdążyć wykonać projekt w przeznaczonym do tego czasie, językiem wybranym do implementacji jest Java. Ustaliliśmy, że aby zmniejszyć prawdopodobieństwo wystąpienia niespójności związanej z wykorzystywaniem różnych narzędzi, aplikacja będzie pisana przy pomocy Eclipse IDE. 

\subsection{Praca w zespole}
    Praca grupowa była wspomagana poprzez wykorzystanie sytemu kontroli wersji GIT, oraz poprzez utorzenie Google Doc'a projektu. W trakcie realizacji przeprowadziliśmy również kilka spotkań mających na celu, kontrolę przebiegu prac, wyjaśnienie watpliwości / niejasności wykrytych w trakcie iteracji, a także przydział kolejnych zadań.

\section{Opis logiki}

\section{Algorytm kukułki}

\section{Obsługa aplikacji}

\section{Rezultaty}


\begin{itemize}

\end{document} 